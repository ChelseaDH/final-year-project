\documentclass[11pt,a4paper,british,twocolumn]{bhamarticle}
\usepackage[utf8]{inputenc}
\usepackage{multicol}
\usepackage{float}
\usepackage{babel}
\usepackage{csquotes}
\usepackage{isodate}
\usepackage[usenames, dvipsnames]{color,xcolor}
\usepackage{listings}

\usepackage[hidelinks]{hyperref}
\hypersetup{colorlinks, linkcolor={red!50!black}, citecolor={blue!50!black}, urlcolor={blue!80!black}}

% Configure bibliography
\usepackage[backend=biber,sorting=none,citestyle=numeric-comp]{biblatex}
\addbibresource{references.bib}

\usepackage{report}

% chktex-file 36

% Tikz for diagrams
\usepackage{tikz}
\usetikzlibrary{calc,shapes,arrows.meta,positioning}
\tikzstyle{box}   = [rectangle, minimum height=1.5em, minimum width=4em, draw=black!50, thick]
\usepackage{bytefield}

\title{A TCP/IP networking stack implementation in userland}
\author{Joe Groocock -- 1467414} % chktex 8
\supervisor{Dr Ian Batten}
\department{Computer Science}
\degree{MSc in Computer Science}

\begin{document}

\maketitle

\tableofcontents
\newpage

% TODO: Make abstract relevant to final project
% - it states the problem that you set out to solve
% - it describes your solution and method
% - it states a conclusion about the success of the solution
\begin{abstract}
    TCP/IP network stacks included in most modern operating systems are opaque to the user and mostly unconfigurable. This article aims to design and produce a coherent specification for an efficient, high performance TCP/IP network stack that can be used as a software library, standalone application or system-wide daemon to provide network access to many applications. It is provided as a learning and academic research tool that can be tweaked, extended and analysed in depth to aid in gaining a better understanding of how operating systems handle networking and TCP/IP as whole.
\end{abstract}

%! TEX root=../report.tex

\section{Introduction}
% Describe the problem to be solved

    \subsection{Background}
    % Explain enough background to understand the problem

    % Clearly state project aims
    % Provide an overview of structure of solution?
    % Explain what is in each section of this document

    %! TEX root = ../report.tex

\section{Internet Protocol Overview}\label{sec:intro-ip}
The Internet Protocol, or as it is more commonly known, IP, provides global addressing across networks. IP mandates that every interconnected host must have an IP address to be able to communicate. It also ensures that any interconnected hosts can communicate with any other interconnected host, given thier address through the process of routing, regardless of the physical distance or barriers. The largest IP network of all is the `internet' where any device with an address can (in theory) communicate with any other device, anywhere across the planet.

% Talk about IPv4 vs IPv6
To date there have been two prominent versions of IP, both of which are still in widespread used today across the internet and private intranets alike. Version 4 of IP, first published in \citeyear{cerf1974protocol} by \citeauthor{cerf1974protocol}~\cite{cerf1974protocol}, the first non-experimental datagram protocol in the Internet Suite. It is still the most widely used protocol today almost 45 years on.

Following the massive success of IPv4 is version 6 of the IP suite. Drafted as a standard in 1998 to replace IPv4 and to resolve the impending IPv4 address exhaustion problem boasting significantly more usable host addresses and many smaller incremental improvements over the previous version 4. Only reaching maturity as an official `Internet Standard' in mid 2017, acceptance and adoption has been slow but is picking up much needed traction to eventually match and finally overtake the dated and problematic IPv4 in coming years.

        % Compare addressing schemes
    \subsection{Terminology}
    \todo{Should there be a glossary for this?}

    \subsection{Addressing}
    IPv4 packets allocate 32 bits for host addresses, for each source and destination. Given every combination of those values, it results in roughly 4.3 million address ($4.3 \times 10^{9}$ \textit{OR} $2^{32}$). Due to reserved groups and lack of foresight during the design phase, only about 3.7 million of those addresses are usable due to reserved allocation such as those defined in RFC 1918 and other wastage of various kinds.

    IPv4 implements `Network Address Translation' (NAT) which is an ugly hack to compensate for the ever-growing demand for addresses, even after there are no more available. NAT works by translating one or more public facing addresses to many private addresses, bidirectionally. It has been twisted and contorted as it has evolved to allow many clients make simultaneous outgoing connections from behind the same IP address and even accept incoming connections given the correct configuration.

    IPv6 \textit{addressess} the major caveat in version 4 of the protocol with much larger address values of 128 bits which results in roughly 340 undecillion ($3.4 \times 10^{38}$ \textit{OR} $2^{128}$) usable addresses. The available addressable space is so vast that currently only a small portion is available for public use and the smallest division that can be allocated by a provider is $2^{64}$ addresses.

    \subsection{Subnets}
    Subnets are divisions of the whole IP space, useful for allocating pieces to the entities that own rights to them. Traditionally, subnets were defined `classfully' where the value of the address defined how large the network it belonged to was. Primarily there were three classes of varying sizes:
    \begin{itemize}
        \item{Class A occupied half of the IPv4 space ranging from \texttt{0.0.0.0/8} to \texttt{128.0.0.0/8}, consisting of 256 networks of $2^{24}$ hosts each.}
        \item{Class B defined $16,384$ networks, ranging from \texttt{128.0.0.0/16} to \texttt{191.0.0.0/16}, each with $65,536\ (2^{16})$ hosts.}
        \item{Class C boasted $2,097,152\ (2^{21})$ networks, each with only $256\ (2^{8})$ hosts.}
    \end{itemize}

    It was originally planned to delegate various network segments to companies based on their requirements. Large enterprises such as DARPA, Ford, HP and Apple, at the time, were allocated a class A network whilst smaller companies were allowed class B and C portions.

    Introduced in 1993, `Classless Inter-Domain Routing' (CIDR) was introduced to permit more flexible network separation with the use of a \textit{variable-length subnet mask}. CIDR is denoted it two forms, either a binary representation such as \texttt{255.255.255.0} or as a decimal value preceded with a slash \texttt{xxx.xxx.xxx.xxx/24}. The value represented by the CIDR refers to the amount of bits that define the `network' portion of the address. The remainder of the bits in the address refer to the hosts in that network.

    CIDR has for the most part replaced classful networks and is used in both IPv4 and IPv6 for defining network subdivisions.

        \subsubsection{Subnetting example}
        Given the typical example host address \texttt{192.168.8.99/24}, similar to one that may be found in consumer hardware used in a small home network, the following information can be extrapolated:
        \begin{itemize}[noitemsep]
            \item{The host address is \texttt{192.168.8.99}}
            \item{The `network address' is \texttt{192.168.8.1}}
            \item{The `broadcast address' is \texttt{192.168.8.255}}
            \item{The `subnet mask' is \texttt{255.255.255.0}}
            \item{There are 254 \textit{usable} host addresses}
            \item{Hosts between \texttt{192.168.8.1} and \texttt{192.168.8.254}, inclusive, can be communicated with directly, without the need for a router/gateway}
        \end{itemize}

    % Explain what it is an how it works. interconnected graph, with routers
    \subsection{Routing}
    Communications are exchanged between internet hosts through a process called `routing'. Packets are dispatched from the source and propogate through one router at a time until they arrive at their destination.

    Routes are defined per interface as a combination of an IP address and a `subnet mask' in CIDR form. Packets can be either delivered directly to the destination host, if it resides within one of the local subnets, or via a router.
    In the case of local addresses, the hardware address of the destination host is used for the link-layer protocol, which is ususally ethernet, and the packet is sent directly to that host.
    Alternatively, if there is no local route available, the hardware address of the specified gateway host is used for the link-layer instead and the packet is sent to the gateway, in the hope that it will forward the packet on.

    Routers are internet hosts primarily to forward packets from one network to another. Often routers will know which networks they are neighbouring so can with reasonable accuracy choose a short route for any given packet. In the case of large networks such as the internet, packets will `hop' from one router to the next, following the routes that match the destination IP address until the packets reaches the destination subnet and ultimately the destination host addressed by the packet.

    IP makes no guarantees about packet delivery, instead it operates on a `best effort' policy; all a router can do is attempt to forward the packet on in the hope that it will eventually arrive at it's final destination.

    Routing is used in many different ways depending on the use case. End-user clients such as personal computers attached to small private networks typically are only aware of a single `route', via their local \textit{gateway} which are dynamically assigned when the device connects to the network. A gateway is a router on the local subnet that can forward a packet toward it's final destination by one \textit{hop}.
    Conversely, routers are largely interconnected devices which are permanently connected to many networks simultaneously. Known routes will be statically assigned, either manually or autonomously by a routing protocol, and forward packets between the various subnets, in effect connecting them together.

    %! TEX root = ../report.tex

\section{TCP Overview}
    Modern TCP is a very complex protocol with many extensions and experimental variations. This overview mostly covers the essentials as defined by the original specification, RFC~793~\cite{rfc793}. Extensions or variations are very useful for improving performance and reliability but absolutely not required for a functioning TCP. % chktex 13

    \subsection{Terminology}
        \begin{itemize}
            \item \texttt{SYN}: Synchronise flag\\
                \textit{Set to establish a new connection}
            \item \texttt{FIN}: Finish flag\\
                \textit{Indicate `no more data to send', closing existing connection}
            \item \texttt{ACK}: Acknowledgement flag\\
                \textit{Confirming that data arrived successfully or acknowledging a command}
            \item \texttt{RST}: Reset flag\\
                \textit{Sent in reply to immediately terminate any connection}
            \item Bit field: \textit{Binary data where each individual bit represents a boolean value e.g. 0 or 1}
        \end{itemize}

    \subsection{TCP packet header}
        A TCP packet, or segment, contains an initial header followed by the packet payload. The header can be a variable length but is always at least 20 bytes and at most 60 bytes so small payloads are an inefficient use of a segment as there is always a 20+ byte overhead, per packet. The header is always a multiple of 32 bits (4 bytes) which is the \textit{word size} for the TCP header, and is padded with zeros to fill any remaining space.

        Most of the information carried in the header is metadata for identifying the connection and the data within the packet. Many of the values such as Sequence Number and Window size correspond to the sender; each peer will have their own individual values.

        \bigskip
        \begin{figure}[H]
            \centering
            \fontsize{8pt}{10pt}
            \begin{bytefield}[bitheight=2em,bitwidth=0.028\columnwidth]{32}
                \bitheader{0,4,8,10,12,14,16,24,31} \\ % chktex 8
                \bitbox{16}{Source Port} & \bitbox{16}{Destination Port} \\
                \bitbox{32}{Sequence Number} \\
                \bitbox{32}{Acknowledgement Number} \\
                \bitbox{4}{Data offset} & \bitbox{6}{Reserved} &
                \bitbox{1}{\tiny \rotatebox{90}{URG}} & \bitbox{1}{\tiny \rotatebox{90}{ACK}} &
                \bitbox{1}{\tiny \rotatebox{90}{PSH}} & \bitbox{1}{\tiny \rotatebox{90}{RST}} &
                \bitbox{1}{\tiny \rotatebox{90}{SYN}} & \bitbox{1}{\tiny \rotatebox{90}{FIN}} &
                \bitbox{16}{Window} \\
                \bitbox{16}{Checksum} & \bitbox{16}{Urgent Pointer} \\
                \bitbox{24}{Options} & \bitbox{8}{Padding} \\
                \wordbox[lrt]{1}{Payload~\dots} \\
                \skippedwords[0.8em]{} \\
                \wordbox[t]{1/2}{}
            \end{bytefield}
            \vspace*{-1em}
            \caption{TCP Packet header}\label{fig:tcphdr}
        \end{figure}

        In order of appearance, the fields in the header are as follows:

        \begin{itemize}
            \item \textbf{Source Port} \textit{2 bytes} Port from which this packets was dispatched.
            \item \textbf{Destination Port} \textit{2 bytes} Port number open to receive on the recipient device.
            \item \textbf{Sequence Number} \textit{4 bytes} Sequential count of the first byte in the payload, identifying the sequence position of the segment.
            \item \textbf{Acknowledgement number} \textit{4 bytes} Sequential count of successfully received bytes.
            \item \textbf{Data offset} \textit{4 bits} Size of the TCP header in 32-bit words, as the header is padded to multiples of 4 bytes. This indicates the start of the payload and the end of the header within the packet.
            \item \textbf{Flags} \textit{12 bits} A \textit{bit field} containing the connection control flags.
            \item \textbf{Window} \textit{2 bytes} Size, in bytes, of the receive window available to fill.
            \item \textbf{Checksum} \textit{2 bytes} The \textit{internet checksum}~\cite{internetchecksum} of the entire TCP segment and the upper \textit{Internet Protocol} layer. Used to ensure (probable) integrity of the data from corruption.
            \item \textbf{Urgent Pointer} \textit{2 bytes} Indicates the offset from the \textit{Sequence number} of the last urgent byte when the \texttt{URG} flag is set.
            \item \textbf{Options} \textit{0-40 bytes} Optionally, a series of TCP extension options followed by padding to the nearest 4th byte.% chktex 8
            \item \textbf{Payload} \textit{0-MSS bytes} Sequenced data for the TCP connection. Can be empty. Limited by the \textit{MSS} of the connection.
        \end{itemize}

    \subsection{Connection-oriented}
        One of the defining features of TCP is the famous \textit{three-way handshake}~(Figure~\ref{fig:threeway}) that every connection begins and ends with.

        \begin{figure}[H]
            \begin{tikzpicture}[>=latex,scale=1]
                \footnotesize
                \coordinate (TL) at (1.2,3);
                \coordinate (BL) at (1.2,0);
                \coordinate (TR) at (3.4,3);
                \coordinate (BR) at (3.4,0);
                \coordinate (LA) at ($(TL)!.25!(BL)$);
                \coordinate (LB) at ($(TL)!.65!(BL)$);
                \coordinate (LC) at ($(TL)!.70!(BL)$);
                \coordinate (LD) at ($(TL)!.75!(BL)$);
                \coordinate (RA) at ($(TR)!.14!(BR)$);
                \coordinate (RB) at ($(TR)!.40!(BR)$);
                \coordinate (RC) at ($(TR)!.45!(BR)$);
                \coordinate (RD) at ($(TR)!.50!(BR)$);
                \coordinate (RE) at ($(TR)!.90!(BR)$);
                % Lines & labels
                \draw (TL) node[above]{\large Client};
                \draw (TR) node[above]{\large Server};
                \draw[thick] (TL)--(BL) (TR)--(BR);
                % LISTEN
                \draw (RA) node[right]{%
                    \begin{tabular}{l}
                        \verb$LISTEN$\\
                        \textit{passive open}
                    \end{tabular}
                };
                % \texttt{SYN}
                \draw (LA) node[left]{%
                    \begin{tabular}{r}
                        \textit{active open}\\
                        \verb$SYN_SENT$
                    \end{tabular}
                };
                \draw[->] (LA) -- (RB) node[midway,sloped,above]{\verb$SYN$};
                \draw (RC) node[right]{%
                    \begin{tabular}{l}
                        \verb$SYN_RCVD$\
                    \end{tabular}
                };
                % \texttt{SYN}/\texttt{ACK}
                \draw[->] (RD) -- (LB) node[midway,sloped,above]{\verb$SYN, ACK$};
                \draw (LC) node[left]{%
                    \begin{tabular}{r}
                        \verb$ESTABLISHED$\
                    \end{tabular}
                };
                % \texttt{ACK}
                \draw[->] (LD) -- (RE) node[midway,sloped,above]{\verb$ACK$};
                \draw (RE) node[right]{%
                    \begin{tabular}{l}
                        \verb$ESTABLISHED$\
                    \end{tabular}
                };
            \end{tikzpicture}
            \caption{TCP Three-way handshake}\label{fig:threeway}
        \end{figure}

        A client requests a new connection by sending an initial packet with the \texttt{SYN} bit set, as well as an initial random \textit{Sequence number}. When a packet marked with \texttt{SYN} is received at the server, a new connection is being established and the server replies with a \texttt{SYN}/\texttt{ACK} packet. This reply does one of two things:
        \begin{enumerate}
            \item The mutual \texttt{SYN} flag indicates the connection is being accepted along with a separate random sequence number.
            \item The \texttt{ACK} flag being set along with the \textit{Acknowledgement number} indicates it understood the initial \textit{Sequence number}.
        \end{enumerate}
        The sequence number from the initial flag is reused, incremented by one and sent back as the acknowledgement number in the \texttt{SYN}/\texttt{ACK} packet.

        Finally, the client replies with the sequence number sent by the server, incremented by one, which completes the three-way handshake. At this point the connection is set up and ready to be used to transmit data.

        When the connection is finished with, both endpoints need to mutually close the connection, in a very similar fashion to the \texttt{SYN}, \texttt{SYN}/\texttt{ACK}, \texttt{ACK} pattern.
        Either side of the connection can initiate a \texttt{FIN} at any point, indicating there is no more data to send to which the peer must respond with an \texttt{ACK} of the final sequence number incremented by one to confirm the \texttt{FIN} packet was received.

        Finish packets can behave slightly differently to \texttt{SYN} packets as there is no requirement to send a \texttt{FIN} packet immediately after receiving one. In fact, a connection can remain \textit{half closed} for quite a while whilst one side finishes transmitting data. This is a common pattern for some software applications as only simplex communication is required. It is also acceptable to close a connection quickly with \texttt{FIN}, \texttt{FIN}/\texttt{ACK}, \texttt{ACK}.

        If at any point a \texttt{SYN}, \texttt{FIN} or other packet has not been acknowledged by the peer within a certain time frame, the packet will be resent under the assumption it never arrived; this is part of the reliability of TCP (See Section~\ref{sec:reliable}).

        Following this pattern ensures both peers are in agreement of the state of the connection at all times. If at any point an invalid connection request is sent, to a closed port for example, it is instantly replied to with an \texttt{RST} packet indicating that the client should immediately discard of the connection as it is invalid.

    \subsection{Ordered}
        \textit{Sequence numbers} are used to track each individual byte sent over the connection. For each byte successfully received by the peer, the acknowledgement number is incremented to that value. Each packet contains a sequence number for the first byte of the payload in that segment indicating the position of the data in the continuous stream. If a packet arrives out of order, it can be kept aside until it is required or dropped and it will be retransmitted. Data that has been sent but not acknowledged is assumed to be lost so, after a timeout, it is retransmitted.

        Quite often, especially with small local networks or across short distances, for the most part packets will arrive in order and complete so little retransmission is required, just some reordering and payload reassembly.

        There are various methods of tracking missing packets, storing and reordering packets, indicating that packets are lost to reduce retransmission latency and so on that are used in TCP implementations which can improve performance and throughput in many cases, however, they are not required for basic TCP functionality.

    \subsection{Reliable}\label{sec:reliable}
        Acknowledging data is the first method used by TCP to ensure that all data arrives intact. The sender keeps track of the sequence number of the last acknowledgement it received and also keeps buffered the transmitted but unacknowledged data in the case where it has to be retransmitted. As data is received at the client, acknowledgements are sent in reply confirming the data was received.

        Complementing that, a primitive checksum known as the \textit{Internet Checksum} is used to verify integrity of the data. By modern standards it is a very poor checksum implementation, however, it is very fast to compute and is still sufficient to ensure no serious corruption has occurred. In most cases, even if the checksum is still valid, corruption caused to the header could be detected and the packet will be discarded likely due to out-of-band sequence numbers or similar. The \textit{Internet Checksum} is defined as ``the 16 bit one's complement of the one's complement sum of all 16 bit words in the header and text''~\cite[3.1]{rfc793}.

    \subsection{Extensible}
        TCP has a vast array of extensions, specified in the `Option' section at the end of the packet header, that are used to fix many issues, improve performance and add extra functionality. This section will cover the commonly used options of note, but there are many more that are widely used and some still in development.

            At the advent of TCP, networks were comparatively slow of only a few Megabits per second, however, technology has advanced and now links are available in excess of 100 Gigabits per second. The default \textit{Window size} defined in TCP is only 64KiB which at 100Gbps can be filled in just 5 microseconds which can slow down transmission dramatically as the buffer has to be cleared before it can be filled again. To leverage these blistering transfer speeds, Window Scaling~\cite[2]{rfc1323} was introduced to increase the size of the default receive window. 
        \subsubsection{Window Scaling}\label{sec:wscale}

            Window scaling must be negotiated during the initial handshake of the connection; a client requests wscale in the first \texttt{SYN} packet to which the peer replies with a wscale option in the returning \texttt{SYN}/\texttt{ACK} segment, confirming the choice to use scaling for the remainder of the connection. Should no reply to the wscale option be sent, it is assumed the peer does not support window scaling and therefore the connection must operate without scaling.

            The option specifies a scale value between 0 and 14 which should be applied as a \textit{bitwise left shift} to the window size allowing a maximum receive window size of 1GiB ($65536 << 14$). It is completely valid to negotiate a `zero' shift in one direction to disable scaling for one client.

        \subsubsection{Selective \texttt{ACK}}\label{sec:sack}
            Acknowledgements are part of the reason TCP is so useful, but in particularly poor network conditions, the performance of TCP can suffer from excessive retransmissions, driving latency up and throughput down. Often in lossy transfers only a small amount of packets will be lost but TCP defines that data should be acknowledged until that point and all data afterwards will be retransmitted. It is often that most or all data after a missing packet will have arrived successfully but would be unnecessarily retransmitted.

            Selective \texttt{ACK} aims to reduce and abolish retransmissions for data that has already arrived by indicating to the sender which blocks have or have not arrived. It is an option appended specifically to \texttt{ACK} packets to give more detail about which byte ranges should be retransmitted. The option has a variable payload length to allow inclusion of the left and right boundaries of each block that is being acknowledged.

            \citeauthor{rfc896} proposed in \citeyear{rfc896} the concept of reducing the amount of small packets being transmitted by buffering outgoing data for a small period of time before sending it. This is today known as \textit{Nagle's Algorithm} which is defined to address the ``small packet problem'' by not transmitting new data whilst there is still unacknowledged data in transmission. Sent data is queued indefinitely in the transmit buffer until an acknowledgement is received or until the buffer contains enough data to fill at least one segment.
        \subsubsection{\citeauthor{rfc896}'s Algorithm}

            The segment size in this case is the \textit{Maximum Segment Size} calculated by the lowest MTU in the route between the two communicating nodes, minus the header sizes. MSS tends to be 1460, or less, assuming the standard Ethernet (IEEE 802.3) MTU of 1500 is used.

        \subsubsection{TCP Corking}
            Linux implements a feature typically referred to as \texttt{TCP\_CORK} which is similar in nature to the operation of Nagle's Algorithm. The ultimate goal of corking is the same as Nagle, but is achieved in a different way. Corking will always buffer data, even after the reception of an acknowledgement, until there is enough to fill a segment or until the timeout has passed.  Linux uses a default corking timeout of 200ms.

        \subsubsection{Congestion Control}
            One of the final main features of TCP is the ability to survive and recover from network congestion with the use of complex control algorithms. Congestion control actually refers to a set of tools that a TCP should have, discussed in detail in the following sections:

        \subsubsection{Slow Start}
            A connection between two nodes is limited by the speed of the slowest intermediary node, such as a router, network interface or switch. To avoid overwhelming any of these devices resulting in packet loss, a TCP should assume the worst-case and start transmitting slowly. Congestion controlled TCP implementations maintain \textit{congestion window} which is the representation to the sender of the state of congestion in the network between the connected peers.

            Slow start initialises a small congestion window, usually a small multiple of the MSS\@. Each time data is transmitted, the sender only releases at most the amount specified by the congestion window. For each acknowledgement received for all transmitted data, the congestion window is increased, often by a factor of two. Each subsequent transmission will be larger than the previous as the congestion window increases. When a lost segment is detected, slow start will stop and the standard linear growth Congestion Avoidance algorithm will kick in.

        \subsubsection{Congestion Avoidance}
            There have been many varying implementations of Congestion Avoidance algorithms with ranging feature sets for different scenarios. Some offer fast recovery, smaller loss or improved throughput. The aim of these algorithms is to continue a stable transmission between the two hosts whilst preventing subsequent packet loss by over-sending and overloading the network.

            Popular algorithms for this are: Vegas, BIC, CUBIC, Westwood, Reno, Tahoe and New Reno. Typically `New Reno' is used in Unix systems today as it provides the best performance and stability with minimal changes required to either sender or receiver.



%! TEX root = ../report.tex

\subsection{\mbox{Overview}}
% RFC675 overview
When TCP was originally proposed in December 1974, by \citeauthor{rfc675} in RFC 675~\cite{rfc675}, flow diagrams and implementation suggestions were defined but lacked specific pseudo-code routines that could be implemented directly, were not provided. At that time TCP was young and naive due to having little exposure to real-world use cases. Furthermore, the limitations of the technology of the era were apparent in the original design given the scope and proposed use of the protocol, compared to the significantly higher speed demands in 2017.

% RFC793 overview & level-ip implementation to the spec
RFC 793 followed some years later and provided a more specific breakdown, that can be directly translated and implemented in code, of each event within the protocol and an appropriate procedure to handle it~\cite[Page~54-77]{rfc793}. \citeauthor{rfc793} who published the specification, now recognised as an official internet standard, defined TCP as `a connection-oriented, end-to-end reliable protocol' as it is now widely known. The document provides a well-defined list of requirements for the protocol to run, as well as a multitude of provisions for successful operation within the promises.
Many implementations of the TCP/IP stack including, but not limited to, `Level-IP' by \citeauthor{levelip-spec} follows this specification~\cite[\texttt{src/tcp\_input.c} line~262]{levelip-spec} very closely which, in theory, produces a TCP that should interwork seamlessly, as per the specification, with any other correctly implemented TCP. % chktex 13

% lwIP goals
Not all TCP/IP stacks are created equal; there are numerous incentives for developing alternate implementations, for example `lwIP' from \citeauthor{lwip}~(\citeyear{lwip}) was built `to reduce memory usage and code size, making lwIP suitable for use in small clients with very limited resources such as embedded systems'. There are many inefficiencies in `standard' networking stacks like those included in popular operating systems such as Linux, BSD, macOS and Windows, to name a few, especially regarding memory usage. These protocols make the assumption that the physical system has considerable amounts of memory available for receiving, processing and duplicating network data both in the kernel and in user applications.

\citeauthor{lwip} makes the opposite assumption and as a result produced a system where minimal replication of data and little wasted memory allocation occurs. Using dynamically sized packet buffers \texttt{pbufs}, \citeauthor{lwip} made efficient use of RAM, ROM and pooled memory to address network data without requiring it to be copied to a local storage space before being actioned. Through many small optimisations like these, lwIP was, and still is, a very effective network stack, usable on even the most restricted hardware which in the growing interconnected embedded device market is invaluable.
% TODO: Maybe talk about uIP here, if I have space/time/the will to live

% mTCP goals
Conversely, some alternate implementations exist for quite the opposite reasons such as `mTCP' \citeauthor{jeong2014mtcp}, which was constructed as `a highly scalable user-level TCP stack for multicore systems'. The intention was for mTCP to outperform competing solutions in packet throughput and data volume. According to their claims, mTCP can surpass Linux by a factor of 25 in `small message transactions' while also boosting regular performance of popular applications between 33\% and 320\%. Such performance numbers are impressive, especially considering the widespread use of Linux for commercial applications and hosting, which begs the question: ``Why is it significantly faster than the default Linux implementation and why hasn't Linux caught up yet?''~\cite[2.2, 3]{jeong2014mtcp}

Many of the improvements suggested by \citeauthor{jeong2014mtcp}~\cite{jeong2014mtcp} are very intelligent applications of high-speed network adapters and multicore processor systems, such as servers. Any TCP network stack that is to be run in these kinds of scenarios should consider these optimisations for improved performance. It is likely that many of the proposed enhancements would also benefit low-power single CPU systems too, with the most probable outcome being reduced latency and memory usage.

\subsection{Demultiplexing TCP}\label{sec:demultiplex}
\citeauthor{braun:inria-00074040} designed a modified BSD TCP/IP stack where the IP layer resides in the kernel and TCP is split in two between kernel and userspace into TCPU and TCPK respectively. TCP processing is moved mostly into the user region, residing as local code in the calling process. The only exception to this is the `demultiplexing' step, TCPK, where TCP frames are routed to the correct user program based on the ports and addresses from the IP layer packet, providing security for the receiving process.

% Demultiplexing in userspace, or not
\citeauthor{braun:inria-00074040} theorised that a potential alternative method for demultiplexing packets would be to pass all incoming traffic directly into a userspace daemon for processing, removing the requirement on kernel modifications. However, it is concluded that this concept is impractical and inefficient compared with the alternative solution (above) as packets are processed by two userspace applications, causing more context switches passing data from the daemon to the receiving processes~\cite[2.1]{braun:inria-00074040}\cite[3]{edwards1995experiences}. Generally, this assumption of relative inefficiency is true, however, in certain circumstances there can be situations where this is actually a practical and viable solution. This project aims to implement a usermode-only networking stack, meaning kernel modifications are not plausible and therefore this solution is ideal when optimised appropriately, reducing the context switching overhead.

\subsection{Throughput performance}\label{sec:thruput}
Much of the research surrounding TCP, particularly that providing userspace implementation detail, is focussed on optimising the protocol for high performance and high throughput. \citeauthor{edwards1995experiences}, in \citeyear{edwards1995experiences} demonstrated throughput of 160 Mbit/s using a userspace TCP implementation running over coaxial token-ring ATM networking, making use of the solid existing HP-UX kernel TCP code along with single-copy from the NIC to the user application using shared memory in the TCP stack and the application. This particular arrangement managed 80\% of the performance of the existing single-copy TCP stack and 150\% of the double-copy kernel stack. \citeauthor{braun:inria-00074040} in \citeyear{braun:inria-00074040} yielded similar results of around 40-50\%~\cite[5]{braun:inria-00074040} between the default kernel and TCPU/TCPK stacks running on much less powerful hardware. % chktex 8
% TODO: Talk about lwIP here (2,6,12)
%     Copying from kernel <-> user

% mTCP thread-local storage and multicore
\citeauthor{jeong2014mtcp} shows that with the use of multiple receive queues spread across individual CPU cores and fewer locks along with improved buffer management and fewer context switches between kernel and user mode their implementation can yield a much improved throughput compared to Linux and other user-mode concepts. Much of the improved performance is thanks to the use of \textit{thread local storage} of socket buffers, TCP buffer pools and other data structures relating to individual threads that are not shared. Greatly reduced usage of locks across threads, reducing \textit{Lock contention}, helps to minimise idle time in both incoming and outgoing packet processing.

\subsection{Reducing data copying overhead}\label{sec:reducecopy}
A common theme across many protocol implementations is the focus on reducing overhead due to data copying. Any copied data is potentially wasted time, especially if the same functionality could be programmed using a \textit{zero-copy} approach, where incoming packets are deposited directly from the network adapter into the user program without being duplicated one or more times. Many of the related works include some variation of a transmit/receive queue utilised by every layer from the NIC to the user program, minimal or zero data copying. \citeauthor{jeong2014mtcp} utilised an event-driven packet I/O library to divide incoming batches of packets into multiple queues written directly from the network interface and passed directly through into the user TCP~\cite[3.1]{jeong2014mtcp}.

\citeauthor{braun:inria-00074040} took a different approach to reducing data copying where typically packet buffers are copied twice: once from the device to an input buffer queue then secondly into the application buffer. In certain situations, by the process of \textit{header prediction}, the kernel copies the header data only. Then assuming the predicted header size was correct, enclosed packet data is copied directly into the user program receive buffer~\cite[4.1]{braun:inria-00074040}. By modern standards this is less than optimal, as now there exists suitable methods to skip copying entirely and use memory-mapped buffers.

Due to the limitations of the BSD socket API, \textit{zero-copy} networking is not trivially implemented without some significant changes to how data is copied into the user memory, although there is one solution that harnesses the power of hardware offload. Using a feature of virtual memory to remap memory pages, data can be offloaded by the network interface directly into aligned memory pages, separating header from the payload~\cite[2.3]{chase2001end}. Specific hardware and driver support is required for this, and even more complex connection tracking is required for TCP offload. A result of the memory page mapping is large contiguous blocks of memory containing packet payloads produced without a single data copy.

\subsection{TCP security considerations}
By moving processing from kernel to userspace, protections such as obscurity from other programs is reduced. \citeauthor{braun:inria-00074040} raised the issue that with a userspace TCP process, data for multiple connections is loaded into the memory space of a single process, enabling it to be viewed or modified by any privileged program~\cite[1, 2.1]{braun:inria-00074040}. In most cases this is not a significant issue, however, it should still be considered. The only plausible solution is to move the demultiplexing stage back into the kernel to restore the connection security.

%! TEX root = ../report.tex

\section{Specification}
    % How you analysed the problem
    % Give an appropriate specification of the solution

    % Requirements from journal
    % \begin{itemize}
    %     \item This project must produce a TCP/IP implementation that runs completely in Linux userspace with no/minimal kernel modification
    %     \item This project must follow the original RFC 793~\cite{rfc793} specification for TCP
    %     \item This project must provide a network stack as a library such that it can be easily implemented in other software
    %         \begin{itemize}
    %             \item There should be sufficient documentation to demonstrate examples of how this can be achieved
    %         \end{itemize}
    %     \item This project should provide sufficient interface to the user to view the inner workings of a TCP connection
    %     \item This project could implement modern TCP extensions such as Window Scaling, Selective \texttt{ACK} and Congestion Control.
    % \end{itemize}

    % Define specific behavior or functions
    % what a software system should do
    \subsection{Functional Requirements}

    % A requirement that specifies criteria that can be used to judge the operation of a system, rather than specific behaviors.
    % constraints on how the system should do so
    % elaborates a performance characteristic
    \subsection{Non-functional Requirements}

    % What the user expects the software to be able to do
    \subsection{User Requirements}
        

%! TEX root = ../report.tex

\section{Design}
    \subsection{Layered Architecture}
    \subsection{TCP buffer management}
    \subsection{No specific API requirement}
        % designed in a way that it can sit below any kind of API..
    \subsection{Language choice}
        % c is good
        % rust is probably ok
        % go is bad - Hayo 2k18
    % standard format for returning errors
    \subsection{Error Handling}

%! TEX root = ../report.tex

% A detailed account of the implementation and testing of your software
% Explain the conceptual structure of the algorithms.
% Also explain what data structures you used
    % how the algorithms were implemented
% What implementation decisions did you take, and why?
    % There is no need to list every little function and procedure and explain its working in elaborate detail; use your judgement on what is appropriate to include

\section{Implementation}
\todo{Describe the basic packet-flow operation}

    \subsection{Memory Management}
        % refcounting
        \subsubsection{Reference Counting}

    % logging with macros for verbose logs
    \subsection{Logging}

    \subsection{Standalone Library}
        % netstack_init()
        % netstack_cleanup()
        %

    \subsection{Header Manipulation}

    \subsection{Interfaces}\label{sec:intf}

    \subsection{Internet Protocol}

        \subsubsection{Routing}

    \subsection{\mbox{Address Resolution Protocol}}

    \subsection{TCP}
        \subsubsection{Segment Arrives}
        \subsubsection{Retransmission}
        \subsubsection{Concurrency}

    \subsection{Socket API}
        \subsubsection{Netstack API}
        \subsubsection{BSD Socket API}

    \subsection{Issues}
        \subsubsection{Standards}
        % RFC 793 is incomplete and very vague
        \subsubsection{Threading \& Locking}
        % refcount
        % pthread_cond

        % BSD socket API design
        % non-blocking I/O (poll/select) :(


%! TEX root = ../report.tex

% Explain your testing strategy and provide convincing examples of tests
    % In project involving software development it is usual to relate the tests to your Software Requirements Specification.

\section{Testing}
    % CI building
    % multiple libc impl, multiple compilers
    \subsection{Logging}


%! TEX root = ../report.tex

\section{Project Management}
    % Continuous integration
    % Kanban board with prioritised tasks
    % Too much work, not enough time for one person
        % prioritising only the most important tasks
        % DHCP, UDP, addressing, cmdline utils, configuration files all less important than TCP


%! TEX root = ../report.tex

\section{Results}
    % retransmissions
    % performance
        % with logging


%! TEX root = ../report.tex

\section{Evaluation} % Discussion
    % how it went
    \subsection{Issues that I thought were going to be but weren't actually issues}
        % refer to journal section 5
    % how it could be improved
    \subsection{Improvements}
        % send/recv buffer implementation (recv buffer structure is a better design and should be used for send buffers too)
        \subsubsection{Send/Receive Buffers}

        % parameterised ip/route specification
        \subsubsection{Addressing Configuration}
            % IP address, routes
            % MAC addresses
        % queue'd out-of-order packets to be re-processed for control (FIN), not just for payload
        \subsubsection{Out-of-order TCP queuing}

        \subsubsection{Parallel Processing Efficiency}
            % One thread per _established_ TCP connection
                % Do NOT start a thread for every SYN packet to prevent SYN flood DoS attacks

    % where the project should go from here
    \subsection{Future Work}
        % everything in improvements, obvs
        % TCP extensions for improved performance/throughput & reduced overhead
        % configuration
            % configurable logging
                % disable anything below particular errors
        % system-wide network daemon with IPC library
            % see journal section 4.4
        \subsubsection{Use cases}


%! TEX root = ../report.tex

\chapter{Conclusion}



\printbibliography % chktex 1

\end{document}
