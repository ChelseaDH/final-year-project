\documentclass[11pt,a4paper,british]{bhamarticle}
\usepackage[margin=7em]{geometry}
\usepackage[utf8]{inputenc}
\usepackage{multicol}
\usepackage{float}
\usepackage{babel}
\usepackage{csquotes}
\usepackage{isodate}

\usepackage{tikz}
\usetikzlibrary{calc,arrows}

\usepackage{enumitem}
\setlist[itemize]{leftmargin=*}

\usepackage[
    backend=biber,
    sorting=none,
    citestyle=numeric-comp
]{biblatex}
\usepackage[hidelinks]{hyperref}
\addbibresource{references.bib}

% Fix paragraph spacing and use of noindent
\usepackage[parfill]{parskip}
\usepackage[raggedright]{titlesec}
\setlength{\columnsep}{18pt}
\setlength{\parskip}{1\medskipamount}
\titlespacing{\section}{0pt}{1.4\parskip}{0.2\parskip}
\titlespacing{\subsection}{0pt}{1.4\parskip}{0.2\parskip}
\titlespacing{\subsubsection}{0pt}{1.4\parskip}{0.2\parskip}

\title{A TCP/IP networking stack implementation in userland}
\author{Joe Groocock}
\supervisor{Dr Ian Batten}
\department{Computer Science}
\degree{Computer Science MSci}
\gyear{2017}

\begin{document}
\pagenumbering{gobble}

\maketitle

\begin{abstract}
    % TODO: Write coherent abstract
\end{abstract}

\tableofcontents
\newpage

\pagenumbering{arabic}
\begin{multicols}{2}

\section{Introduction}
Transmission Control Protocol is an essential component of the modern internet used every day on all class of devices. It is the primary component in ensuring data is delivered reliably, in order and error-checked from end-to-end. TCP is often referred to as TCP/IP as it is tightly coupled with the lower layer \textit{Internet Protocol} that provides addressing and packet routing across large networks, like the public internet.

Over the years, TCP has been adapted and upgraded with extensions and variations to make it run more efficiently and intelligently, with improvements such as complex \textit{congestion control} algorithms to prevent systems and routers from getting overwhelmed, \textit{window scaling} to allow fast communicating clients to buffer more than the baseline 64 Kilobyte limit and \textit{Nagle's Algorithm}~\cite{rfc896} responsible for queuing small payloads of data together before sending to reduce multiple successive small packets being dispatched, amongst a long list of others extensions, some of which are defunct or have been replaced.

Every connected operating system contains a TCP/IP implementation from it's vendor, however most are based on the original Berkely TCP implementation from BSD Unix~\cite{rfc793} and have been adapted to work more efficiently in modern workloads. TCP is a solid protocol that is backwards compatible by design so any devices, dating back to the early 1980s, should still be able to communicate with modern systems. TCP implementations in most devices are well-written and have the benefit of almost 40 years of experience and incremental improvements. For general-use computing such as web browsing and downloading files, these improvements are sufficient and have been optimised as such.

Implementations in every major operating system allow for interaction with TCP via an \textit{Application Programming Interface} or API, to communicate with the enclosed network stack, however none of them allow access to view or modify the lower-level inner workings of the protocol. There are some use-cases where being able to interact, preview and utilise TCP can be useful, especially in software development and academic projects where low-level protocol access would be beneficial.

\subsection{Aim}
The primary aim of this project is to provide, as a software library, a callable, interactive TCP/IP stack that can be used standalone, as a system daemon, for use by multiple processes, or to be picked apart and analysed for educational purposes. The code will be provided in a modular way such that parts can be called as desired, or replaced with alternative implementations as to observe the difference between the two.

Multiple similar programs exist that provide many of the same features, but none that are drop-in replacements for the inbuilt kernel stack that can be run without configuration as a userland application without the need for recompilation of the calling process. This project's aim is to provide a very simple user experience, where the network stack can be invoked at runtime to replace the traditional system calls:

\texttt{\$ netstack curl -vv \url{github.com}}

\section{TCP Overview}
Modern TCP is a very complex protocol with many extensions and experimental variations. This overview mostly covers the essentials as defined by the original specification, RFC~793~\cite{rfc793}. Extensions or variations are very useful for improving performance and reliability but absolutely not required for a functioning TCP.

\subsection{Terminology}
\begin{itemize}
    \item \texttt{SYN}: Synchronise\\
        \textit{Set to establish a new connection}
    \item \texttt{FIN}: Finish\\
        \textit{Indicate closure of existing connection}
    \item \texttt{ACK}: Acknowledgement\\
        \textit{Confirming that data arrived successfully or a acknowledging a command}
    \item \texttt{RST}: Reset\\
        \textit{Sent in reply to an invalid connection}
    \item Bit field: \textit{Data where each individual bit represents a boolean value e.g. 0 or 1}
\end{itemize}

\subsection{TCP packet header}
A TCP packet, or segment, contains an initial header followed by the packet payload. The header can be a variable length but is always at least 20 bytes and at most 60 bytes so small payloads are an inefficient use of a segment as there is always a 20+ byte overhead, per packet.

Most of the information carried in the header is metadata for identifying the connection and the data within the packet. Many of the values such as Sequence Number, Window size etc correspond to the sender; each peer will have their own individual values. In order of appearance, the fields in the header are as follows:

\begin{itemize}
    \item \textbf{Source Port} \textit{2 bytes} Port from which this packets was dispatched.
    \item \textbf{Destination Port} \textit{2 bytes} Port number open to receive on the recipient device.
    \item \textbf{Sequence Number} \textit{4 bytes} Sequential count of the first byte in the payload, identifying the sequence position of the segment.
    \item \textbf{Acknowledgement number} \textit{4 bytes} Sequential count of successfully received bytes.
    \item \textbf{Data offset} \textit{4 bits} Size of the TCP header in 32-bit words, as the header is padded to multiples of 4 bytes. This indicates the start of the payload and the end of the header within the packet.
    \item \textbf{Flags} \textit{12 bits} A \textit{bit field} containing the following control flags: \texttt{FIN}, \texttt{SYN}, \texttt{RST}, \texttt{PSH}, \texttt{ACK} and \texttt{URG}.
    \item \textbf{Window} \textit{2 bytes} Size, in bytes, of the receive window available to fill.
    \item \textbf{Checksum} \textit{2 bytes} The \textit{internet checksum}~\cite{internetchecksum} of the entire TCP segment and the upper \textit{Internet Protocol} layer. Used to ensure (probable) integrity of the data from corruption.
    \item \textbf{Urgent Pointer} \textit{2 bytes} Indicates the offset from the \textit{Sequence number} of the last urgent byte when the \texttt{URG} flag is set.
    \item \textbf{Options} \textit{0-40 bytes} Optionally, a series of TCP extension options followed by padding to the nearest 4th byte.
\end{itemize}

%   0                   1                   2                   3
%  0 1 2 3 4 5 6 7 8 9 0 1 2 3 4 5 6 7 8 9 0 1 2 3 4 5 6 7 8 9 0 1
% +-+-+-+-+-+-+-+-+-+-+-+-+-+-+-+-+-+-+-+-+-+-+-+-+-+-+-+-+-+-+-+-+
% |          Source Port          |       Destination Port        |
% +-+-+-+-+-+-+-+-+-+-+-+-+-+-+-+-+-+-+-+-+-+-+-+-+-+-+-+-+-+-+-+-+
% |                        Sequence Number                        |
% +-+-+-+-+-+-+-+-+-+-+-+-+-+-+-+-+-+-+-+-+-+-+-+-+-+-+-+-+-+-+-+-+
% |                    Acknowledgement Number                     |
% +-+-+-+-+-+-+-+-+-+-+-+-+-+-+-+-+-+-+-+-+-+-+-+-+-+-+-+-+-+-+-+-+
% |  Data |           |U|A|P|R|S|F|                               |
% | Offset| Reserved  |R|C|S|S|Y|I|            Window             |
% |       |           |G|K|H|T|N|N|                               |
% +-+-+-+-+-+-+-+-+-+-+-+-+-+-+-+-+-+-+-+-+-+-+-+-+-+-+-+-+-+-+-+-+
% |           Checksum            |         Urgent Pointer        |
% +-+-+-+-+-+-+-+-+-+-+-+-+-+-+-+-+-+-+-+-+-+-+-+-+-+-+-+-+-+-+-+-+
% |                    Options                    |    Padding    |
% +-+-+-+-+-+-+-+-+-+-+-+-+-+-+-+-+-+-+-+-+-+-+-+-+-+-+-+-+-+-+-+-+
% |                             data                              |
% +-+-+-+-+-+-+-+-+-+-+-+-+-+-+-+-+-+-+-+-+-+-+-+-+-+-+-+-+-+-+-+-+

\subsection{Connection oriented}
One of the defining features of TCP is the famous \textit{three-way handshake}~(Figure~\ref{fig:threeway}) that every connection begins and ends with.

\begin{figure}[H]
    \begin{tikzpicture}[>=latex]
        \footnotesize
        \coordinate (TL) at (1.2,3);
        \coordinate (BL) at (1.2,0);
        \coordinate (TR) at (3.4,3);
        \coordinate (BR) at (3.4,0);
        \coordinate (LA) at ($(TL)!.25!(BL)$);
        \coordinate (LB) at ($(TL)!.65!(BL)$);
        \coordinate (LC) at ($(TL)!.70!(BL)$);
        \coordinate (LD) at ($(TL)!.75!(BL)$);
        \coordinate (RA) at ($(TR)!.14!(BR)$);
        \coordinate (RB) at ($(TR)!.40!(BR)$);
        \coordinate (RC) at ($(TR)!.45!(BR)$);
        \coordinate (RD) at ($(TR)!.50!(BR)$);
        \coordinate (RE) at ($(TR)!.90!(BR)$);
        % Lines & labels
        \draw (TL) node[above]{\large Client};
        \draw (TR) node[above]{\large Server};
        \draw[thick] (TL)--(BL) (TR)--(BR);
        % LISTEN
        \draw (RA) node[right]{%
            \begin{tabular}{l}
                \verb$LISTEN$\\
                \textit{passive open}
            \end{tabular}
        };
        % \texttt{SYN}
        \draw (LA) node[left]{%
            \begin{tabular}{r}
                \textit{active open}\\
                \verb$SYN_SENT$
            \end{tabular}
        };
        \draw[->] (LA) -- (RB) node[midway,sloped,above]{\verb$SYN$};
        \draw (RC) node[right]{%
            \begin{tabular}{l}
                \verb$SYN_RCVD$\
            \end{tabular}
        };
        % \texttt{SYN}/\texttt{ACK}
        \draw[->] (RD) -- (LB) node[midway,sloped,above]{\verb$SYN, ACK$};
        \draw (LC) node[left]{%
            \begin{tabular}{r}
                \verb$ESTABLISHED$\
            \end{tabular}
        };
        % \texttt{ACK}
        \draw[->] (LD) -- (RE) node[midway,sloped,above]{\verb$ACK$};
        \draw (RE) node[right]{%
            \begin{tabular}{l}
                \verb$ESTABLISHED$\
            \end{tabular}
        };
    \end{tikzpicture}
    \caption{Three-way handshake}
    \label{fig:threeway}
\end{figure}

A client requests a new connection by sending an initial packet with the \texttt{SYN} bit set as well as an initial random \textit{Sequence number}. When a packet marked with \texttt{SYN} is received at the server, a new connection is being established and the server replies with a \texttt{SYN}/\texttt{ACK} packet. This reply does one of two things:
\begin{enumerate}
    \item The mutual \texttt{SYN} flag indicates the connection is being accepted along with a separate random sequence number.
    \item The \texttt{ACK} flag being set along with the \textit{Acknowledgement number} indicates it understood the initial \textit{Sequence number}.
\end{enumerate}
The sequence number from the initial flag is reused, incremented by one and sent back as the acknowledgement number in the \texttt{SYN}/\texttt{ACK} packet.

Finally, the client replies with the sequence number sent by the server, incremented by one, which completes the three-way handshake. At this point the connection is set up and ready to be used to transmit data.

When the connection is finished with, both endpoints need to mutually close the connection, in a very similar fashion to the \texttt{SYN}, \texttt{SYN}/\texttt{ACK}, \texttt{ACK} pattern.
Either side of the connection can initiate a \texttt{FIN} at any point, indicating there is no more data to send to which the peer must respond with an \texttt{ACK} of the final sequence number incremented by one to confirm the \texttt{FIN} packet was received.

Finish packets can behave slightly differently to \texttt{SYN} packets as there is no requirement to send a \texttt{FIN} packet immediately after receiving one. In fact, a connection can remain \textit{half closed} for quite a while whilst one side finishes transmitting data. This is a common pattern for some software applications as only simplex communication is required. It is also acceptable to close a connection quickly with \texttt{FIN}, \texttt{FIN}/\texttt{ACK}, \texttt{ACK}.

If at any point a \texttt{SYN}, \texttt{FIN} or other packet has not been acknowledged by the peer within a certain time frame, the packet will be resent under the assumption it never arrived; this is part of the reliability of TCP (See Section~\ref{sec:reliable}).

Following this pattern ensures both peers are in agreement of the state of the connection at all times. If at any point an invalid connection request is sent, to a closed port, for example, a \texttt{RST} packet is immediately in reply indicating the client should immediately discard of the connection as it is invalid.
\subsection{Ordered}
\textit{Sequence numbers} are used to track each individual byte sent over the connection. For each byte successfully received by the peer the acknowledgement number is incremented to that value. Each packet contains a sequence number for the first byte of the payload in that segment indicating the position of the data in the continuous stream. If a packet arrives out of order, it can be kept aside until it is required or dropped and it will be retransmitted. Data that has been sent but not acknowledged is assumed to be lost so after a timeout it is retransmitted.

Quite often, especially with small local networks or across short distances, for the most part packets will arrive in order and complete so little retransmission is required, just some reordering and payload reassembly.

There are various methods of tracking missing packets, storing and reordering packets, indicating that packets are lost to reduce retransmission latency and so on that are used in TCP implementations which can improve performance and throughput in many cases however they are not required for basic TCP functionality.

\subsection{Reliable}\label{sec:reliable}
Acknowledging data is the first method used by TCP to ensure all data arrives intact. The sender keeps track of the sequence number of the lack acknowledgment it received and also keeps buffered the transmitted but unacknowledged data in the case where it has to be retransmitted. As data is received at the client, acknowledgements are sent in reply confirming the data was received.

Complementing that, a primitive checksum known as the \textit{Internet checksum} is used to verify integrity of the data. By modern standards it is a very poor checksum implementation however it is very fast to compute and is still sufficient to ensure no serious corruption has occurred. In most cases, even if the checksum is still valid, corruption caused to the header will be picked up and the packet will be discarded anyway likely due to out-of-band sequence numbers or similar. The \textit{Internet checksum} is defined as `the 16 bit one's complement of the one's complement sum of all 16 bit words in the header and text'~\cite[3.1]{rfc793}.

\subsection{Extensible}
TCP has a vast array of extensions, specified in the `Option' field in the packet header, that are used to fix many issues, improve performance and add extra functionality. This section will cover the commonly used options of note but there are many more that are widely used and some still in development.

\subsubsection{Window Scaling}
At the advent of TCP, networks were comparatively slow of only a few Megabits per second however technology has advanced and now links are available in excess of 100 Gigabits per second. The default \textit{Window size} defined in TCP is only 64KiB which at 100Gbps can be filled in just 5 microseconds which can slow down transmission dramatically as the buffer has to be cleared before it can be filled again. To leverage these blistering transfer speeds, Window Scaling~\cite[2]{rfc1323} was introduced to increase the size of the default receive window. 

The option specifies a scale value between 0 and 14 which should be applied as a \textit{bitwise left shift} to the window size allowing a maximum receive window size of 1GiB ($65536 << 14$). It is completely valid to negotiate a `zero' shift in one direction to disable scaling for one client.
\subsection{Connection states}

% TODO: Fix tenses (to past, mostly)
\section{Literature Review}
\subsection{\mbox{A brief look at related works}}
% RFC675 overview
When TCP was originally proposed in December 1974, by \citeauthor{rfc675} in RFC 675~\cite{rfc675}, flow diagrams and implementation suggestions were defined but lacked specific pseudo-code routines that could be implemented directly, were not provided. At that time TCP was young and naive due to having little exposure to real-world use cases. Furthermore, the limitations of the technology of the era were apparent in the original design given the scope and proposed use of the protocol, compared to the significantly higher speed demands in 2017.

% RFC793 overview & level-ip implementation to the spec
RFC 793 followed some years later and provided a more specific breakdown, that can be directly translated and implemented in code, of each event within the protocol and an appropriate procedure to handle it~\cite[Page~54-77]{rfc793}. \citeauthor{rfc793} who published the specification, now recognised as an official internet standard, defined TCP as `a connection-oriented, end-to-end reliable protocol' as it is now widely known. The document provides a well-defined list of requirements for the protocol to run, as well as a multitude of provisions for successful operation within the promises.
Many implementations of the TCP/IP stack including, but not limited to, `Level-IP' by \citeauthor{levelip-spec} follows this specification~\cite[\texttt{src/tcp\_input.c} line~262]{levelip-spec} very closely which, in theory, produces a TCP that should interwork seamlessly, as per the specification, with any other correctly implemented TCP. % chktex 13

% lwIP goals
Not all TCP/IP stacks are created equal; there are numerous incentives for developing alternate implementations, for example `lwIP' from \citeauthor{lwip}~(\citeyear{lwip}) was built `to reduce memory usage and code size, making lwIP suitable for use in small clients with very limited resources such as embedded systems'. There are many inefficiencies in `standard' networking stacks like those included in popular operating systems such as Linux, BSD, macOS and Windows, to name a few, especially regarding memory usage. These protocols make the assumption that the physical system has considerable amounts of memory available for receiving, processing and duplicating network data both in the kernel and in user applications. \citeauthor{lwip} makes the opposite assumption and as a result produced a system where minimal replication of data and little wasted memory allocation occurs. Using dynamically sized packet buffers \texttt{pbufs}, \citeauthor{lwip} made efficient use of RAM, ROM and pooled memory to address network data without requiring it to be copied to a local storage space before being actioned. Through many small optimisations like these, lwIP was, and still is, a very effective network stack, usable on even the most restricted hardware which in the growing interconnected embedded device market is invaluable.
% TODO: Maybe talk about uIP here, if I have space/time/the will to live

% mTCP goals
Conversely, some alternate implementations exist for quite the opposite reasons such as `mTCP' \citeauthor{jeong2014mtcp}, which was constructed as `a highly scalable user-level TCP stack for multicore systems'. The intention was for mTCP to outperform competing solutions in packet throughput and data volume. According to their claims, mTCP can surpass Linux by a factor of 25 in `small message transactions' while also boosting regular performance of popular applications between 33\% and 320\%. Such performance numbers are impressive, especially considering the widespread use of Linux for commercial applications and hosting, which begs the question: ``Why is it significantly faster than the default Linux implementation and why hasn't Linux caught up yet?''~\cite[2.2, 3]{jeong2014mtcp}

Many of the improvements suggested by \citeauthor{jeong2014mtcp}~\cite{jeong2014mtcp} are very intelligent applications of high speed network adapters and multicore processor systems, such as servers. Any TCP network stack that is to be run in these kinds of scenarios should consider these optimisations for improved performance. It is likely that many of the proposed enhancements would also benefit low-power single CPU systems too, with the most probable outcome being reduced latency and memory usage.

\subsection{Demultiplexing TCP}
\citeauthor{braun:inria-00074040} designed a modified BSD TCP/IP stack where the IP layer resides in the kernel and TCP is split in two between kernel and userspace into TCPU and TCPK respectively. TCP processing is moved mostly into the user region, residing as local code in the calling process. The only exception to this is the `demultiplexing' step, TCPK, where TCP frames are routed to the correct user program based on the ports and addresses from the IP layer packet, providing security for the receiving process.

% Demultiplexing in userspace, or not
\citeauthor{braun:inria-00074040} theorised that a potential alternative method for demultiplexing packets would be to pass all incoming traffic directly into a userspace daemon for processing, removing the requirement on kernel modifications. However, it is concluded that this concept is impractical and inefficient compared with the alternative solution (above) as packets are processed by two userspace applications, causing more context switches passing data from the daemon to the receiving processes~\cite[2.1]{braun:inria-00074040}\cite[3]{edwards1995experiences}. Generally, this assumption of relative inefficiency is true, however, in certain circumstances there can be situations where this is actually a practical and viable solution. This project aims to implement a usermode-only networking stack, meaning kernel modifications are not plausible and therefore this solution is ideal when optimised appropriately, reducing the context switching overhead.

\subsection{Throughput performance}
Much of the research surrounding TCP, particularly that providing userspace implementation detail, is focussed on optimising the protocol for high performance and high throughput. \citeauthor{edwards1995experiences}, in \citeyear{edwards1995experiences} demonstrated throughput of 160 Mbit/s using a userspace TCP implementation running over coaxial token-ring ATM networking, making use of the solid existing HP-UX kernel TCP code along with single-copy from the NIC to the user application using shared memory in the TCP stack and the application. This particular arrangement managed 80\% of the performance of the existing single-copy TCP stack and 150\% of the double-copy kernel stack. \citeauthor{braun:inria-00074040} in \citeyear{braun:inria-00074040} yielded similar results of around 40-50\%~\cite[5]{braun:inria-00074040} between the default kernel and TCPU/TCPK stacks running on much less powerful hardware.
% TODO: Talk about lwIP here (2,6,12)
%     Copying from kernel <-> user

% mTCP thread-local storage and multicore
\citeauthor{jeong2014mtcp} shows that with the use of multiple receive queues spread across individual CPU cores and fewer locks along with improved buffer management and fewer context switches between kernel and user mode their implementation can yield a much improved throughput compared to Linux and other user-mode concepts. Much of the improved performance is thanks to the use of \textit{thread local storage} of socket buffers, TCP buffer pools and other data structures relating to individual threads that are not shared. Greatly reduced usage of locks across threads, reducing \textit{Lock contention}, helps to minimise idle time in both incoming and outgoing packet processing.

\subsection{Reducing data copying overhead}
A common theme across many protocol implementations is the focus on reducing overhead due to data copying. Any copied data is potential wasted time, especially if the same functionality could be programmed using a \textit{zero-copy} approach, where incoming packets are deposited directly from the network adapter into the user program without being duplicated one or more times. Many of the related works include some variation of a transmit/receive queue utilised by every layer from the NIC to the user program, minimal or zero data copying. \citeauthor{jeong2014mtcp} utilised an event-driven packet I/O library to divide incoming batches of packets into multiple queues written directly from the network interface and passed directly through into the user TCP~\cite[3.1]{jeong2014mtcp}.

\citeauthor{braun:inria-00074040} took a different approach to reducing data copying where typically packet buffers are copied twice: once from the device to an input buffer queue then secondly into the application buffer. In certain situations, by the process of \textit{header prediction}, the kernel copies the header data only. Then assuming the predicted header size was correct, enclosed packet data is copied directly into the user program receive buffer~\cite[4.1]{braun:inria-00074040}. By modern standards this is less than optimal, as now there exists suitable methods to skip copying entirely and use memory-mapped buffers.

Due to the limitations of the BSD socket API, \textit{zero-copy} networking is not trivially implemented without some significant changes to how data is copied into the user memory, although there is one solution that harnesses the power of hardware offload. Using a feature of virtual memory to remap memory pages, data can be offloaded by the network interface directly into aligned memory pages, separating header from the payload~\cite[2.3]{chase2001end}. Specific hardware and driver support is required for this, and even more complex connection tracking is required for TCP offload. A result of the memory page mapping is large contiguous blocks of memory containing packet payloads produced without a single data copy.

\subsection{\mbox{TCP security considerations}}
By moving processing from kernel to userspace, protections such as obscurity from other programs is reduced. \citeauthor{braun:inria-00074040} raise the issue that with a userspace TCP process, data for multiple connections is loaded into the memory space of a single process enabling it to be viewed or modified by any privileged program~\cite[1, 2.1]{braun:inria-00074040}. In most cases this is not a significant issue however it should still be considered. The only plausible solution is to move the demultiplexing stage back into the kernel to restore the connection security.

% Comparison of kernel vs userspace
%   Problems with each
%       (lwIP 4)
%   Limitations of kernel
%     (mtcp 2.1)
%   Limitations of userspace
%     High-precision timers
%       (mtcp 3.2.2)
%       (RR-2650 4.2)
%       (lwIP 5)
%   Half-half implementations
%     (HP 4.4.8)
%     (RR-2650 1, 2)

\section{Implementation Detail}
% List 'user' requirements
% Produce a specification

% Attaining packets
% Demultiplexing packets
% TCP timers

% Why do we use the Linux kernel's TCP stack?
% https://jvns.ca/blog/2016/06/30/why-do-we-use-the-linux-kernels-tcp-stack/
% https://github.com/lkl/linux
% https://news.ycombinator.com/item?id=12021195

% Features I would like:
%   - Multiple interfaces: tpacket, rawsock, rawinet, tun, tap
%     - Defaults to tpacket, whichever version is supported as it
%       is quickest, fallback to rawsock
%   - Running as a system daemon or a standalone hook-in
%     See also: https://github.com/lkl/linux/blob/master/tools/lkl/lib/hijack/hijack.c
%     - Standalone will use libdl to override libc calls
%       like socket(), recv() and listen() etc
%       - Limitations on file descriptor and port number clashes
%         - Use fds > FD_SETSIZE (1024)? Maybe ulimit will mess with us here
%     - System daemon will connect to other programs either
%       via compiled-in support through libnetstack or another
%       binary that uses libdl and hooks syscalls, like standalone
%   - Options:
%     See also: https://tools.ietf.org/html/rfc7414
%     - Window Scaling is a must. Performance relies on it
%     - SACK
%     - Timestamps?


\section{Implementation Issues}
\subsection{Hijacking the BSD \texttt{socket()} API calls} % chktex 36
% Allocating file descriptors such that they don't clash with standard kernel fds

\subsection{Frames larger than the interface MTU}
% Caused by 'segmentation offloading'
% Affects TCP checksums on packets greater than MSS/MTU

\subsection{Fighting the Linux TCP/IP stack RST packets}
% Using iptables or libiptc to drop RST on specific ports
%  - Port can't be used by the kernel and could clash?
%  - outside of /proc/sys/net/ipv4/ip_local_port_range
% Opening the socket to accept the connection
%  - Eventually socket will timeout
%  - Doesn't block any packets from actually getting out
% Kernel module/extension to prevent RST being generated
%  - Could be quicker but defeats isn't truly userspace

\subsection{Efficiently sending and receiving packets through the kernel interface}
\cite{tpacket} % chktex 2
% "it requires one system call to capture each packet, it requires two if you want to get packet's timestamp"
% Use a zero-copy interface like Linux tpacket_v{1,2,3}
%   BSD socket API doesn't allow for zero-copy as it has no notion of packet buffers and payload segmentation
% Communicating data between daemon and user process without copy (SHM?)

% Signal handling, such as SIGSTP breaking timing

\printbibliography % chktex 1
\end{multicols}
\end{document}
